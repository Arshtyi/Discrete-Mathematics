\newpage
\chapter{基本结构}
\thispagestyle{fancy}
\section{集合}
\subsection{幂集}
\begin{formal}
\begin{definition}[幂集的定义]\label{def:幂集的定义}
    设集合$S$,则$S$的幂集是由$S$的所有子集构成的集合,记作$2^S$或$\powerset{S}$.
\end{definition}
\end{formal}
\begin{formal}
    \begin{theorem}[幂集的大小]\label{thm:幂集的大小}
        \[
        \#\left(
        \powerset{S}
        \right)=2^{\#S}
        \]
    \end{theorem}
\end{formal}
\begin{formal}
    \begin{theorem}[幂集的相等与集合的相等]\label{thm:幂集的相等与集合的相等}
        设集合$S$和$T$,则$S=T$当且仅当$\powerset{S}=\powerset{T}$.
    \end{theorem}
\end{formal}
\subsection{真值集和量词}
\begin{formal}
    \begin{definition}[全称量化与存在量化]\label{def:全称量化与存在量化}
        设谓词$P(x)$和集合$S$,则\begin{enumerate}[label={\textup{(\arabic*)}}]
            \item $\forall x\in S\left(
                P\left(x\right)
            \right)$表示$P\left(x\right)$在集合$S$上的全称量化即是$\forall x\left(
                x\in S\rightarrow P\left(x\right)
            \right)$的简记
            \item $\exists x\in S\left(
                P\left(x\right)
            \right)$表示$P\left(x\right)$在集合$S$上的存在量化即是$\exists x\left(
                x\in S\wedge P\left(x\right)
            \right)$的简记
        \end{enumerate}
    \end{definition}
\end{formal}
\begin{brown}
\begin{example}
    $\forall x\in\bb{R}\left(
        x^2\geqslant0
    \right)$声称对于任意实数$x$,都有$x^2\geqslant0$,这是真的.$\exists x\in\bb{Z}\left(
        x^2=1
    \right)$声称存在整数$x$使得$x^2=1$,这也是真的.
\end{example}
\end{brown}
\begin{formal}
    \begin{definition}[真值集的定义]\label{def:真值集的定义}
        给定谓词$P$和论域$D$,则$P\left(x\right)$的真值集为\[
        \left\{
            x\in D\mid P\left(x\right)
        \right\}
        \]
    \end{definition}
\end{formal}
\begin{brown}
    \begin{example}
        $R\left(x\right)=\left|x\right|=x$的真值集为$\bb{N}$.
    \end{example}
\end{brown}
\subsection{多重集}
\begin{formal}
    \begin{definition}[多重集的定义]\label{def:多重集的定义}
        设$S$是一个包含有重复元素的集合,记作\[
        \left\{
            a_1,a_1,\cdots,a_1;a_2,a_2,\cdots,a_2;\cdots;a_r,a_r,\cdots,a_r
        \right\}
        \]或者\[
        m_1\cdot a_1,m_2\cdot a_2,\cdots,m_r\cdot a_r
        \]其中$m_i$是$a_i$的重复数(不在集合中的元素的重复数为$0$).
    \end{definition}
\end{formal}
\begin{formal}
    \begin{proposition}[多重集的运算]\label{prop:多重集的运算}
        设多重集$P=\left\{
            m_1\cdot a_1,m_2\cdot a_2,\cdots,m_r\cdot a_r
        \right\}$和$Q=\left\{
            n_1\cdot a_1,n_2\cdot a_2,\cdots,n_r\cdot a_r
        \right\}$,则\begin{enumerate}[label={\textup{(\arabic*)}}]
            \item $
            P\cup Q=\left\{
                \max\left(
                    m_1,n_1
                \right)\cdot a_1,\max\left(
                    m_2,n_2
                \right)\cdot a_2,\cdots,\max\left(
                    m_r,n_r
                \right)\cdot a_r
            \right\}
            $
            \item $
            P\cap Q=\left\{
                \min\left(
                    m_1,n_1
                \right)\cdot a_1,\min\left(
                    m_2,n_2
                \right)\cdot a_2,\cdots,\min\left(
                    m_r,n_r
                \right)\cdot a_r
            \right\}
            $
            \item $ 
            P-Q=\left\{
                \max\left(
                    m_1-n_1,0
                \right)\cdot a_1,\max\left(
                    m_2-n_2,0
                \right)\cdot a_2,\cdots,\max\left(
                    m_r-n_r,0
                \right)\cdot a_r
            \right\}
            $
            \item $
            P+Q=\left\{
                \left(
                    m_1+n_1
                \right)\cdot a_1,\left(
                    m_2+n_2
                \right)\cdot a_2,\cdots,\left(
                    m_r+n_r
                \right)\cdot a_r
            \right\}
            $
        \end{enumerate}
    \end{proposition}
\end{formal}
\begin{red}
\begin{remark}
    从\cref{prop:多重集的运算}可以看出,多重集的和在普通集合中退化为集合的并.
\end{remark}
\end{red}