\documentclass[12pt,a4paper,UTF8,twoside]{ctexbook}
\usepackage[tbtags]{amsmath}
\usepackage{extarrows}
\usepackage{amsthm}
\usepackage{amssymb}
\usepackage{appendix}
\usepackage{arydshln}
\usepackage{bm}
\usepackage{cases}
\usepackage[strict]{changepage}
\usepackage{enumerate}
\usepackage{enumitem}
\usepackage{environ}
\usepackage{etoolbox}
\usepackage{hyperref}
\usepackage[nameinlink]{cleveref}
\usepackage{framed}
\usepackage{geometry}
\usepackage{graphicx}
\usepackage{lipsum}
\usepackage{listings}
\usepackage{lmodern}
\usepackage{mathrsfs}
\usepackage{rotating}
\usepackage{mathtools}
\usepackage{multirow}
\usepackage{newtxtext}
\usepackage{fancyhdr}
\usepackage{threeparttable}
\usepackage{tikz-cd}
\usepackage{stmaryrd}
\usepackage[dvipsnames,svgnames]{xcolor}
\usepackage[format=hang,font=small,textfont=it]{caption}
\geometry{top=25.4mm,bottom=25.4mm,left=20mm,right=20mm,headheight=2.17cm,headsep=4mm,footskip=12mm}
\hypersetup{
    colorlinks=false,
    linkcolor=blue,
    citecolor=green,
    urlcolor=red,
}
\setenumerate[1]{itemsep=5pt,partopsep=0pt,parsep=\parskip,topsep=5pt}
\setitemize[1]{itemsep=5pt,partopsep=0pt,parsep=\parskip,topsep=5pt}
\setdescription{itemsep=5pt,partopsep=0pt,parsep=\parskip,topsep=5pt}
\allowdisplaybreaks[4]
\title{\textbf{离散数学笔记}}
\author{机笑}
\date{\today}
\linespread{1.5}
\setcounter{tocdepth}{1}
\definecolor{formalshade}{rgb}{0.95,0.95,1}
\definecolor{greenshade}{rgb}{0.90,0.99,0.91}
\definecolor{redshade}{rgb}{1.00,0.90,0.90}
\definecolor{brownshade}{rgb}{0.99,0.97,0.93}
\newenvironment{formal}{%
\def\FrameCommand{%
\hspace{1pt}%
{\color{DarkBlue}\vrule width 2pt}%
{\color{formalshade}\vrule width 4pt}%
\colorbox{formalshade}%
}%
\MakeFramed{\advance\hsize-\width\FrameRestore}%
\noindent\hspace{-4.55pt}%
\begin{adjustwidth}{}{7pt}%
\vspace{2pt}\vspace{2pt}%
}%
{%
\vspace{2pt}\end{adjustwidth}\endMakeFramed%
}%
\newenvironment{green}{%
\def\FrameCommand{%
\hspace{1pt}%
{\color{Green}\vrule width 2pt}%
{\color{greenshade}\vrule width 4pt}%
\colorbox{greenshade}%
}%
\MakeFramed{\advance\hsize-\width\FrameRestore}%
\noindent\hspace{-4.55pt}%
\begin{adjustwidth}{}{7pt}%
\vspace{2pt}\vspace{2pt}%
}%
{%
\vspace{2pt}\end{adjustwidth}\endMakeFramed%
}%
\newenvironment{red}{%
\def\FrameCommand{%
\hspace{1pt}%
{\color{LightCoral}\vrule width 2pt}%
{\color{redshade}\vrule width 4pt}%
\colorbox{redshade}%
}%
\MakeFramed{\advance\hsize-\width\FrameRestore}%
\noindent\hspace{-4.55pt}%
\begin{adjustwidth}{}{7pt}%
\vspace{2pt}\vspace{2pt}%
}%
{%
\vspace{2pt}\end{adjustwidth}\endMakeFramed%
}%
\newenvironment{brown}{%
\def\FrameCommand{%
\hspace{1pt}%
{\color{BurlyWood}\vrule width 2pt}%
{\color{brownshade}\vrule width 4pt}%
\colorbox{brownshade}%
}%
\MakeFramed{\advance\hsize-\width\FrameRestore}%
\noindent\hspace{-4.55pt}%
\begin{adjustwidth}{}{7pt}%
\vspace{2pt}\vspace{2pt}%
}%
{%
\vspace{2pt}\end{adjustwidth}\endMakeFramed%
}%
\newtheoremstyle{mydefinition}{3pt}{3pt}{\itshape}{0cm}{\bf}{}{1em}{}
\theoremstyle{mydefinition}
%formal
\newtheorem{theorem}{\indent Theorem}[section]
\newtheorem{definition}{\indent Definition}[section]
\newtheorem{proposition}{\indent Proposition}[section]
\newtheorem{criterion}{\indent Criterion}[section]
%green
\newtheorem{lemma}{\indent Lemma}[section]
\newtheorem{corollary}{\indent Corollary}[section]
%brown
\newtheorem{example}{\indent Example}[section]
\newtheorem{problem}{\indent Problem}[section]
\newtheorem*{solution}{\indent Solution}
\newenvironment{Proof}{\begin{proof}[\textnormal{\textbf{Proof}}]}{\end{proof}}
%red
\newtheorem*{remark}{\indent \color{red}{Remark}}
\crefformat{theorem}{\textup{\textbf{Theorem~#2#1#3~}}}
\crefname{theorem}{Theorem}{Theorems}
\crefformat{definition}{\textup{\textbf{Definition~#2#1#3~}}}
\crefname{definition}{Definition}{Definitions}
\crefformat{proposition}{\textup{\textbf{Proposition~#2#1#3~}}}
\crefname{proposition}{Proposition}{Propositions}
\crefformat{criterion}{\textup{\textbf{Criterion~#2#1#3~}}}
\crefname{criterion}{Criterion}{Criteria}
\crefformat{lemma}{\textup{\textbf{Lemma~#2#1#3~}}}
\crefname{lemma}{Lemma}{Lemmas}
\crefformat{corollary}{\textup{\textbf{Corollary~#2#1#3~}}}
\crefname{corollary}{Corollary}{Corollaries}
\crefformat{example}{\textup{\textbf{Example~#2#1#3~}}}
\crefname{example}{Example}{Examples}
\crefformat{problem}{\textup{\textbf{Problem~#2#1#3~}}}
\crefname{problem}{Problem}{Problems}
\lstset{
	alsolanguage = Java,
	alsolanguage = [ANSI]C,    
    alsolanguage = Python,
	alsolanguage = matlab,
	alsolanguage = XML,
	tabsize=4, 
	frame=shadowbox, 
	columns=fullflexible,
	commentstyle=\color[RGB]{0,0.6,0},
	rulesepcolor=\color{red!20!green!20!blue!20},
	keywordstyle=\color{blue!90}\bfseries, 
	showstringspaces=false,
	stringstyle=\color[RGB]{0.58,0,0.82}\ttfamily, 
	keepspaces=true, 
	breakindent=22pt, 
	numbers=left,
	stepnumber=1,
	numberstyle={\color{black}\tiny} ,
	numbersep=8pt,
	basicstyle=\footnotesize\ttfamily,
	showspaces=false, 
	flexiblecolumns=true, 
	breaklines=true, 
	breakautoindent=true,
	breakindent=4em, 
	aboveskip=1em, 
	tabsize=2,
	showstringspaces=false, 
	backgroundcolor=\color{white},   
	escapeinside={\%*}{*)}, 
	fontadjust,
	captionpos=b,
	framextopmargin=2pt,framexbottommargin=2pt,abovecaptionskip=-3pt,belowcaptionskip=3pt,
	xleftmargin=4em,xrightmargin=4em, 
	texcl=true,
	extendedchars=false,columns=flexible,mathescape=true
}
\newcommand{\iddots}{\begin{rotate}{90}$\ddots$\end{rotate}}
\newcommand{\dashline}{\text{---}}
\newcommand{\rmi}{\ensuremath{\mathrm{i}}}
\newcommand{\rme}{\ensuremath{\mathrm{e}}}
\newcommand{\powerset}[1]{\ensuremath{\mathcal{P}\left(#1\right)}}
\newcommand{\bb}[1]{\ensuremath{\mathbb{#1}}}
\begin{document}
\thispagestyle{empty}
\maketitle
\thispagestyle{empty}
\newpage
\section*{前言}
\thispagestyle{empty}
这是作者自己的离散数学笔记,
主要参考教材为机械工业出版社《离散数学及其应用(第八版)》和山东大学自编离散数学教材,欢迎任何有价值的问题探讨与交流.

作者全网名称一般是机笑或者Arshtyi,同时欢迎关注作者的微信公众号:机笑理运.作者的常用联系方式包括 QQ:640006128.

本文并不适合完全意义上的初学者使用,如果必要,可以参考其他资料进行理解.

本文完全使用\LaTeX 写作,所有的\LaTeX 源代码可在Github上作者的仓库找到并下载,项目地址:https://github.com/Arshtyi/Discrete-Mathematics ,包括配套的习题解答也可以在此找到,允许个人预习、复习等其他个人用途或转载但禁止未经作者本人同意的商业相关用途.
\begin{flushright}
    \begin{tabular}{c}
        机笑/Arshtyi\\
        \today \\ 穷诸
        玄辩,若一毫置
        于太虚;\\ 竭世枢
        机,似一滴投于巨壑.
    \end{tabular}
\end{flushright}
\thispagestyle{empty}
\newpage
\pagenumbering{Roman}
\setcounter{page}{1}
\pagestyle{plain}
\tableofcontents
\newpage
\setcounter{page}{1}
\pagenumbering{arabic}
\pagestyle{fancy}
\fancyhf{}
\fancyhead[LO,RE]{\rightmark}
\fancyfoot[LO,RE]{\thepage}
\fancyfoot[C]{By \textit{机笑理运/Arshtyi}}
\newpage
\chapter{基本结构}
\thispagestyle{fancy}
\section{集合}
\subsection{幂集}
\begin{formal}
\begin{definition}[幂集的定义]\label{def:幂集的定义}
    设集合$S$,则$S$的幂集是由$S$的所有子集构成的集合,记作$2^S$或$\powerset{S}$.
\end{definition}
\end{formal}
\begin{formal}
    \begin{theorem}[幂集的大小]\label{thm:幂集的大小}
        \[
        \#\left(
        \powerset{S}
        \right)=2^{\#S}
        \]
    \end{theorem}
\end{formal}
\begin{formal}
    \begin{theorem}[幂集的相等与集合的相等]\label{thm:幂集的相等与集合的相等}
        设集合$S$和$T$,则$S=T$当且仅当$\powerset{S}=\powerset{T}$.
    \end{theorem}
\end{formal}
\subsection{真值集和量词}
\begin{formal}
    \begin{definition}[全称量化与存在量化]\label{def:全称量化与存在量化}
        设谓词$P(x)$和集合$S$,则\begin{enumerate}[label={\textup{(\arabic*)}}]
            \item $\forall x\in S\left(
                P\left(x\right)
            \right)$表示$P\left(x\right)$在集合$S$上的全称量化即是$\forall x\left(
                x\in S\rightarrow P\left(x\right)
            \right)$的简记
            \item $\exists x\in S\left(
                P\left(x\right)
            \right)$表示$P\left(x\right)$在集合$S$上的存在量化即是$\exists x\left(
                x\in S\wedge P\left(x\right)
            \right)$的简记
        \end{enumerate}
    \end{definition}
\end{formal}
\begin{brown}
\begin{example}
    $\forall x\in\bb{R}\left(
        x^2\geqslant0
    \right)$声称对于任意实数$x$,都有$x^2\geqslant0$,这是真的.$\exists x\in\bb{Z}\left(
        x^2=1
    \right)$声称存在整数$x$使得$x^2=1$,这也是真的.
\end{example}
\end{brown}
\begin{formal}
    \begin{definition}[真值集的定义]\label{def:真值集的定义}
        给定谓词$P$和论域$D$,则$P\left(x\right)$的真值集为\[
        \left\{
            x\in D\mid P\left(x\right)
        \right\}
        \]
    \end{definition}
\end{formal}
\begin{brown}
    \begin{example}
        $R\left(x\right)=\left|x\right|=x$的真值集为$\bb{N}$.
    \end{example}
\end{brown}
\subsection{多重集}
\begin{formal}
    \begin{definition}[多重集的定义]\label{def:多重集的定义}
        设$S$是一个包含有重复元素的集合,记作\[
        \left\{
            a_1,a_1,\cdots,a_1;a_2,a_2,\cdots,a_2;\cdots;a_r,a_r,\cdots,a_r
        \right\}
        \]或者\[
        m_1\cdot a_1,m_2\cdot a_2,\cdots,m_r\cdot a_r
        \]其中$m_i$是$a_i$的重复数(不在集合中的元素的重复数为$0$).
    \end{definition}
\end{formal}
\begin{formal}
    \begin{proposition}[多重集的运算]\label{prop:多重集的运算}
        设多重集$P=\left\{
            m_1\cdot a_1,m_2\cdot a_2,\cdots,m_r\cdot a_r
        \right\}$和$Q=\left\{
            n_1\cdot a_1,n_2\cdot a_2,\cdots,n_r\cdot a_r
        \right\}$,则\begin{enumerate}[label={\textup{(\arabic*)}}]
            \item $
            P\cup Q=\left\{
                \max\left(
                    m_1,n_1
                \right)\cdot a_1,\max\left(
                    m_2,n_2
                \right)\cdot a_2,\cdots,\max\left(
                    m_r,n_r
                \right)\cdot a_r
            \right\}
            $
            \item $
            P\cap Q=\left\{
                \min\left(
                    m_1,n_1
                \right)\cdot a_1,\min\left(
                    m_2,n_2
                \right)\cdot a_2,\cdots,\min\left(
                    m_r,n_r
                \right)\cdot a_r
            \right\}
            $
            \item $ 
            P-Q=\left\{
                \max\left(
                    m_1-n_1,0
                \right)\cdot a_1,\max\left(
                    m_2-n_2,0
                \right)\cdot a_2,\cdots,\max\left(
                    m_r-n_r,0
                \right)\cdot a_r
            \right\}
            $
            \item $
            P+Q=\left\{
                \left(
                    m_1+n_1
                \right)\cdot a_1,\left(
                    m_2+n_2
                \right)\cdot a_2,\cdots,\left(
                    m_r+n_r
                \right)\cdot a_r
            \right\}
            $
        \end{enumerate}
    \end{proposition}
\end{formal}
\begin{red}
\begin{remark}
    从\cref{prop:多重集的运算}可以看出,多重集的和在普通集合中退化为集合的并.
\end{remark}
\end{red}
\newpage
\chapter{关系}
\thispagestyle{fancy}
\newpage
\pagenumbering{Roman}
\setcounter{page}{1}
\pagestyle{plain}
\nocite{*}
\bibliographystyle{plain}
\clearpage
\phantomsection
\addcontentsline{toc}{chapter}{参考文献}
\bibliography{References}
\end{document}